\documentclass[11pt]{article}
\usepackage{titling}
\usepackage{polski}
\usepackage[utf8]{inputenc}
\usepackage{fontspec}
\usepackage{setspace}
\usepackage{hyperref}
\usepackage{cite}

\setmainfont{Calibri Light}
\setstretch{1.15}
\setlength{\droptitle}{-10em}

\newenvironment{itquote}
  {\begin{quote} \begin{center}\itshape}
  {\end{center}  \end{quote}   \ignorespacesafterend}


\setlength{\parindent}{0pt}
\setlength{\oddsidemargin}{-15pt}
\setlength{\textwidth}{480pt}
\setlength{\textheight}{600pt}

\author{Michał Puchyr}
\date{}
\title{\textbf{Czy dzisiejsza Polska wspiera przedsiębiorczość małych i średnich przedsiębiorstw?}}

\begin{document}
\maketitle
\section*{Wprowadzenie}
Celem artykułu jest analiza działań podejmowanych przez Polskę w ostatnich latach w wspieraniu działalności małych i średnich przedsiębiorstw.
Duży nacisk zostanie położony na przedstawienie jakie wsparcie dla przedsiębiorców oferuje Polska w 2023 roku 
oraz jakie zmiany w tym zakresie nastąpiły w ostatnich latach.

\section*{Obecna sytuacja gospodarcza}
Polska gospodarka jest szóstą co do wielkości gospodarką w Unii Europejskiej i zarazem największą wśród członków Unii Europejskiej z krajów byłego bloku wschodniego.
W 2019 roku Polska gospodarka rozwijała się w tempie 4,1\% rocznie, co było najlepszym wynikiem w całej Unii Europejskiej.
Przez ONZ jest uznawana za kraj wysoko rozwinięty pod względem wskaźnika rozwoju społecznego HDI który 
bierze pod uwagę takie czynniki jak długość życia, średnią długość edukacji odbytej
przez 25-latków i oczekiwany czas edukacji dzieci w wieku szkolnym, jak również realny PKB per capita, tj. z uwzględnieniem siły nabywczej.

Na sytuację gospodarczą w roku 2022 wpłynęła pandemia COVID-19 która spowodowała spowolnienie gospodarcze w całej Unii Europejskiej oraz na świecie.
Również rosyjska agresja na Ukrainę oraz kryzys energetyczny spowolniły rozwój gospodarczy w Polsce.
2022 rok był pierwszym od 1991 roku w którym Polska gospodarka skurczyła się ale mimo to Polska gospodarka jest jedną z najszybciej rozwijających się gospodarek w Europie.


\section*{Małe i średnie przedsiębiorstwa - definicja}
Definicja Małych i Średnich Przedsiębiorstw (dalej nazywana jako MŚP) została ustalona w 17 czerwca 2014 roku przez Komisję Europejską i jest stosowana w całej Unii Europejskiej.

Definicja przedsiębiorstwa według Komisji Europejskiej \cite{Komisja}

\begin{itquote}
    ZAŁĄCZNIK I

    \textbf{DEFINICJA MŚP}

    Artykuł 1

    \textbf{Przedsiębiorstwo}

    Za przedsiębiorstwo uważa się podmiot prowadzący działalność gospodarczą bez względu na jego formę prawną.
    Zalicza się tu w szczególności osoby prowadzące działalność na własny rachunek oraz firmy rodzinne zajmujące się rzemiosłem lub inną działalnością,
    a także spółki lub stowarzyszenia prowadzące regularną działalność gospodarczą.
    
    Artykuł 2

    \textbf{Pułapy zatrudnienia oraz pułapy finansowe określające kategorię przedsiębiorstwa}

    1. Do kategorii mikroprzedsiębiorstw oraz małych i średnich przedsiębiorstw („MŚP”) należą przedsiębiorstwa,
    które zatrudniają mniej niż 250 pracowników i których roczny obrót nie przekracza 50 milionów EUR, lub roczna
    suma bilansowa nie przekracza 43 milionów EUR.
\end{itquote}

\section*{Rola MŚP w polskiej gospodarce}

Sektor przedsiębiorstw wytwarza blisko trzy czwarte wartości PKB (71,6\%). Największy udział w tworzeniu
PKB mają mikroprzedsiębiorstwa - około 29,5\%. Cały sektor MSP generował 43,6\% PKB (dane za 2020 r.)\cite{RaportPARPoMSP}.
\medskip

Analiza udziału przedsiębiorstw w tworzeniu PKB ze względu na sektor gospodarki pokazuje istotne różnice
pomiędzy dużymi przedsiębiorstwami a MSP. W przypadku MSP największe znaczenie miał sektor Usług, którego udział w tworzeniu PKB wyniósł
w 2020 r. 46,8\%, podczas gdy w dużych firmach - 30,1\%.

Drugi w kolejności był Handel (23,5\% - MSP; 17,8\% - duże
firmy). Z kolei w dużych przedsiębiorstwach widocznie większy wkład w tworzenie PKB
w porównaniu z sektorem MSP m Przemysł (48,0\% - duże firmy; 17,8\% - MSP),
najmniejszy zaś Budownictwo (3,9\% - duże firmy; 11,8\% - MSP) \cite{RaportPARPoMSP}. 

\section*{Sposoby wspierania MŚP w Polsce}

\subsection*{Kredyty preferencyjne}

Jednym ze sposobów wspierania MŚP w Polsce są udzielane przez państwo kredyty preferencyjne.
Według Głównego Urzędu Statystycznego jest to udzielanie kredytów na określone rodzaje działalności, 
korzystniejszych wobec określonej grupy kredytobiorców pod względem ogólnych warunków umowy, oprocentowania, 
harmonogramu spłat lub innych warunków kredytowania (np. możliwość zastosowania karencji w spłacie kapitału, 
stosowania prolongaty w spłacie części albo całości zadłużenia kapitału lub odsetek) \cite{DefKredytPref}.

\medskip

W ostatnich latach kredyty preferencyjne były udzielane przez Bank Gospodarstwa Krajowego (dalej nazywany jako BGK).
Jednym z przykładów programów udzielania kredytów przez BGK jest program pożyczek płynnościowych z 
Programu Operacyjnego Inteligentny Rozwój dla mikro, małych i średnich firm.
Ilość pieniędzy przeznaczonych na ten projekt wynosi ponad 4 mld zł. Celem projektu jest wsparcie dla przedsiębiorców z sektora MŚP, 
którzy z powodu pandemii COVID-19 znaleźli się w trudnej sytuacji finansowej oraz także dla firm, które mają problemy przez skutki
rosyjskiej agresji wobec Ukrainy \cite{BgkProgramPlynnosci}.


\pagebreak

\bibliography{bibliography}{}
\bibliographystyle{unsrt}

\end{document}