\documentclass[12pt]{article}
\usepackage[a4paper, total={6in, 8in}]{geometry}
\usepackage{titling}

\setlength{\droptitle}{-10em}

\setlength{\parindent}{0pt}
\setlength{\oddsidemargin}{-15pt}
\setlength{\textwidth}{480pt}
\setlength{\textheight}{730pt}

\author{Michał Puchyr}
\title{\textbf{Aftermath of Polish parliamentary elections 2023}}

\begin{document}
\maketitle

\subsection*{Results}

On the 15th october 2023 in Poland there was a parliamentary election.
These elections will go down in history as those with the biggest attendance in history of \textbf{Third Republic of Poland} which was 74\%.

The winner of the election is \textbf{right-wing populist} party Law and Justice (PiS) with 35,38\% of votes.
The second place was taken by the Civic Coalition (KO) with 30,70\% of votes.
Further places were taken respectively by Third Road 14,40\%, New Left 8,61\% and Confederation 7,16\%.

\subsection*{President nomination}

On the 6th of november 2023 polish president Andrzej Duda has nominated current prime minister Mateusz Morawiecki for the role of possible prime minister in
the upcoming goverment. This decision was not a surprise as president Duda started in president election as a candidate from Law and Justice. Although he is not a member of the party
he is still seen as a person who is close to the party. The argumentation for this decision was the fact that PiS won the election and there was a tradition that 
the president gives the mission of forming a goverment to the representant of the winning party.

\subsection*{What now?}

Now prime minister Mateusz Morawiecki has 14 days to form a goverment. It is expected that he will not be able to do that as the all parties from opposition 
has already announced that they will not support idea of goverment formed by Law and Justice. The only way for PiS to form a goverment is to convince some of the opposition members.
Even as Matuesz Morawiecki announced in interview for Interia.pl that he can see himself as a minister in goverment of PSL leader Władysław Kosiniak-Kamysz who clearly denies
that possibility.

\subsection*{Most possible result}

Although Law and Justice has the most mandates they won't be able to form a goverment. 
Andrzej Duda's decision is seen as a elongation of the process of ceding power to the opposition.
This move might be seen as a prelude to long fight between the president and the parliament.

\end{document}