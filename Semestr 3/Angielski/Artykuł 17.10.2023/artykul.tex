\documentclass[12pt]{article}
\usepackage[a4paper, total={6in, 8in}]{geometry}
\usepackage{titling}

\setlength{\droptitle}{-10em}

\setlength{\parindent}{0pt}
\setlength{\oddsidemargin}{-15pt}
\setlength{\textwidth}{480pt}
\setlength{\textheight}{730pt}

\author{Michał Puchyr}
\title{\textbf{Polish parliamentary elections 2023}}

\begin{document}
\maketitle

\subsection*{Results}

On the 15th october 2023 in Poland there was a parliamentary election.
These elections will go down in history as those with the biggest attendance in history of \textbf{Third Republic of Poland} which was 74\%.

The winner of the election is \textbf{right-wing populist} party Law and Justice (PiS) with 35,38\% of votes.
The second place was taken by the Civic Coalition (KO) with 30,70\% of votes.
Further places were taken respectively by Third Road 14,40\%, New Left 8,61\% and Confederation 7,16\%.

\subsection*{Biggest surprises}

The biggest surprise is the result of Third Road, New Left and Confederation. As for the first two biggest parties there was no doubt, that these
two will have the highest result. The results for PiS and KO mostly matched the \textbf{polls} before the election. For third party which was Third Road it was
a positive surprise as there was a possibility that they won't get into the parliament. In Poland there is a 5\% \textbf{electoral threshold} for parties and 8\% for coalitions.
As Third Road was starting as coalition consisting of the Polska 2050 and PSL their electoral threshold was 8\%. The result of 14,40\% is a big success for them as their predictions were around from 7\% to 10\%.

\smallskip

The result of New Left and Confederation are seen as defeat because both of these parties were predicted to get around 10\% of votes. It is believed that the reason for this fact is 
high attendance which works negatively for these two parties as their voters are usually more radical and are more likely to vote than the average voter.

\subsection*{Possible goverment}

Although the winner is Law and Justice it does not mean that they will be able to form a \textbf{goverment}.
For now it is expected that the goverment will be formed by the coalition of KO, Third Road and New Left known also as democratic opposition.
In total these three parties combined have got 248 mandates which guarantees them absolute majority in parliament.

For now talks among opposition are ongoing on who will be the prime minister. Soon we will know the answer to this question.

\subsection*{Upcoming problems}

Even if the opposition will form a goverment it will not be easy for them to rule the country.
The main reason for that is the fact that the president of Poland is still Andrzej Duda from Law and Justice.
It is expected that the president will veto many laws passed by the parliament.
To \textbf{override} the veto the parliament will need 3/5 majority which is 276 mandates.
The opposition has only 248 mandates so they will need to convince some of the opposition members to vote for the law.

\end{document}